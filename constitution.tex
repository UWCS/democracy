\documentclass[a4paper,11pt,parskip=half-]{scrartcl} % scrartcl for Montserrat

\input{common/common.tex}

\title{University of Warwick Computing Society Constitution}
\date{29/04/2025}

\begin{document}

\maketitle

\section{Name}
\begin{enumerate}
    \item The name of the society shall be Warwick Students' Union University of Warwick Computing Society.
\end{enumerate}

\section{Aims and Objectives}
\begin{enumerate}
    \item The Society shall have written statement of aims and objectives, providing a clear understanding of the society. This shall be subject to review annually by the University of Warwick Computing Society Executive Committee.
    \item The society aims and objectives shall be:
    \begin{enumerate}
        \item To act as the academic student society for Computer Science and related areas.
        \item To help people become more computer literate.
        \item To provide an environment for like-minded people to collaborate on projects and events.
        \item To offer a number of workshops and academic talks to help students of all subjects learn how to improve their programming abilities.
    \end{enumerate}
    \item The Society is affiliated with the British Computing Society, The Chartered Institute for IT.
    \item The Society, its Executives, its funds and all its activities shall be subject to the provisions of the By-Laws, Regulations and Policy of the University of Warwick Students' Union.
    \item The Society shall be subject to a disciplinary code as laid out by the Students' Union and administered by the Societies Executive.
    \item Any alterations to the Society Constitution must be ratified by the Societies Executive. A provisional copy must be sent to the Societies Officer for approval before the new constitution may take effect.
    \item If the Society has issues arising not mentioned in a personalised Constitution, then this document will become the default. Any issues may be dealt with by contacting your Societies Coordinator.
\end{enumerate}

\section{Memberships}
\begin{enumerate}
    \item Memberships of the Society shall be open to all full, associate, and honorary members of the Students' Union upon payment of the required Societies Federation subscription.
    \item Memberships of the Society are to be renewed on the 1st September of every year.
    \item Only standard members of the society shall be entitled to vote in elections, provided they have joined the society and paid the appropriate subscription fee no less than 14 days before voting begins.
    \begin{enumerate}
        \item The subscription fee to join the society is £0.00.
    \end{enumerate}
    \item The following shall not be entitled to hold the office of an Executive position in the society:
    \begin{enumerate}
        \item An Associate or Honorary member of the Students' Union or Societies Federation.
        \item Any person who has received payment for the provision of services to the society (not including reimbursement of personal expenditure on behalf of the society).
    \end{enumerate}
    \item The Society must have a minimum of 30 members by the sixth week of term one. If the minimum membership is not met, a meeting with the Society, Societies Coordinator and Societies Officer will be scheduled to discuss the future of the Society.
    \item It is the responsibility of the Society Executive Committee to ensure that those attending their events are members of the Union.
\end{enumerate}

\section{Executive Committee}
\begin{enumerate}
    \item The Society's Executive Committee (which may also be referred to as the ``Motherboard'') shall be made up of at least three voting officers of whom two shall be the President/Chair and the Treasurer respectively.
    \item The Executive Committee shall be responsible for the day to day running of the society and may decide upon any matter that has not yet been decided upon by the General Meeting. The Executive Committee shall further be responsible for:
    \begin{enumerate}
        \item Organising the activities of the Society in such a way as to include the greatest possible number of Society members.
        \item Managing the expenditure of the Society's funds in a responsible fashion and in line with the aims, objectives and planned activities of the Society whilst adhering to the SU's financial regulations.
        \item Formulating and submitting an annual bid for funds from the Societies Executive prior to any specified deadline which shall include a statement of activities and objectives for the coming year and detailed justification of the figures contained in the bid.
        \item Formulating and submitting any additional bids for funds from the Societies Executive or groups within the Students' Union.
        \item Assisting any review of the Society's activities and use of funds carried out by a standing committee or group of the Students' Union that has granted funds to the Society.
        \item Upholding the Constitution of the Society and ensuring that its aims and objectives reflect the Society activities.
        \item Ensuring that all society activity abides by the By-laws, regulations, and policy of Warwick SU.
        \item Executive Committee members must attend assigned training to ensure they are equipped to organise the Society.
    \end{enumerate}

    \itemb{Core Officers}
    \begin{enumerate}[series=core]
        \item The Core Officer Duties shall include:
        \begin{enumerate}
            \item To attend the Society's Annual General Meeting.
            \item To attend Societies Council and complete mandatory training sessions/courses.
        \end{enumerate}
% Break out of enumerate for indent
    \end{enumerate}
\end{enumerate}
\doublelistbreak{The core officers shall be:}
% Dummy \item[] {} are required for enum directly inside enum, the listbreak command counteracts the added spacing
\begin{enumerate}[resume]
    \item[] {}
    \begin{enumerate}[resume=core]
	\item[] {}
        \begin{role}{President / Chair / Club Captain / Lord/Lady/Liege/etc Chancellor of the Computers}
            \item The President should organise and oversee the running of The Society.
            \item The President should chair committee meetings.
            \item The President should produce an annual report.
            \item The Executive Committee should appoint a standing President to oversee the President's duties in the event of extended absence or resignation of the President. This position will only be held for a maximum of 10 weeks, until the next Extraordinary or Annual General Meeting.
        \end{role}
        \begin{role}{Treasurer}
            \item The Treasurer should be responsible for the finances of the Society.
            \item The Treasurer should maintain an up-to-date record of their group account in addition to the record kept by the SU finance office.
            \item All funds should be held and processed through the groups Students' Union bank account. No money should be held in personal bank accounts.
            \item The Treasurer should submit grant funding applications.
        \end{role}
        \begin{role}{Welfare Officer}
            \item The Welfare Officer should work confidentially with both society members and exec to prioritise their wellbeing and safety, be available as the first point of contact, and where appropriate, signpost to external services such as 	Wellbeing Services.
            \item The Welfare Officer is not a therapist and should not be expected to act as such.
            \item The Welfare Officer should, as appropriate, inform event organiser(s), the President, or sources of authority within the university/SU, where issues require greater intervention (for example, where members’ safety is at direct risk), or where there are issues such as a conflict of interest.
            \item The Welfare Officer should engage with learning materials such as those provided by (but not strictly limited to) the Active Bystander course and Report \& Support.
            \item For non-sober events, at least one (1) Welfare Officer should aim to act as “sober exec” when in attendance, and failing this for any reason, should at least endeavour to ensure that there are “sober exec” at the event that they can liaise with where needed.
        \end{role}
    \end{enumerate}
    \itemb{Additional Officers}
    \begin{enumerate}
        \begin{role}{Academic Coordinator}
            \item The Academic Coordinator should organise the academic events, such as external guest speakers, presentations on areas relating to Computer Science and workshops.
            \item The Academic Coordinator should act as the point of contact for organisation of events with external speakers, university departments, or other students.
        \end{role}
        \begin{role}{Events Officer}
            \item The Events Officer should be responsible for co-ordinating major society events, e.g. the Computing Ball.
            \item The Events Officer should ensure all event forms for society events are submitted to the Students' Union where necessary and in the required timeframes, such as the event planning packs and external speaker forms.
        \end{role}
        \begin{role}{Freshers' Representative}
            \item The Freshers' Representative should endeavour to facilitate and drive 1st year engagement and events within the society.
            \item The Freshers' Representative should act as a point of contact between the Executive Committee and 1st year students.
            \item Up to one (1) Freshers' Representative should be selected for each category of events the society runs, currently: Academic, Social, Gender Inclusivity, and Treasurer. Up to two (2) for Gaming and Technical.
            \item The Fresher's Representative for each category should assist with and be supported by their relevant executive members in the planning and running of events they perceive as relevant to their year, along with the general activities of the society.
            \item This office will be elected in a by-election run between weeks one to five of term one.
        \end{role}
        \begin{role}{Gaming Coordinator}
            \item The Gaming Coordinator should organise all gaming-related events.
        \end{role}
        \begin{role}{Gender Inclusivity Officer}
            \item This officer should endeavour to help create and maintain a welcoming atmosphere for marginalised genders in the society.
            \item This officer should act as a point of contact for organizing events aimed to promote initiatives beneficial towards underrepresented demographics in Computer Science and liaise with the department in this capacity.
        \end{role}
        \begin{role}{Publicity Officer}
            \item The Publicity Officer is responsible for reviewing and posting announcements to secondary social media platforms.
            \item The Publicity Officer is responsible for creating graphics as needed for society announcements and operation.
        \end{role}
        \begin{role}{Secretary}
            \item The Secretary should act as the point of contact for general enquiries with the society.
            \item The Secretary should take minutes of every Society Executive Committee meeting and publish them on the Society website.
            \item The Secretary should organise all clothing and merchandise orders for members of the society.
        \end{role}
        \begin{role}{Social Secretary}
            \item The Social Secretary should organise the social events run by the society each term.
            \item The Social Secretary should act as the convenor with other society executive committees for the purpose of organising a collaborative social.
            \item The Social Secretary should ensure the safety and welfare of all members at any and all Society-run socials.
        \end{role}
        \begin{role}{Sports Officer}
            \item The Sports Officer should organise sporting events for the society.
            \item The Sports Officer should involve the society in campus leagues.
        \end{role}
        \begin{role}{Technical Officer}
            \item The Technical Officer should oversee the running of the digital and physical services offered by the Society.
            \item The Technical Officer should disclose any planned maintenance periods for society services at least 7 days before maintenance is to begin.
        \end{role}
    \end{enumerate}
\end{enumerate}

\section{Meetings}
\begin{enumerate}
    \item The Executive must meet at least five times per term to ensure the Society is operated to a high standard.
    \item The Society Executive shall call at least one General Meeting per year for the purposes of discussing plans and activities for the coming year. This must be held by Week 10, Term 2. This GM may also be used to hold Society Officer elections.
    \item The Society Executive shall give at least seven days' notice of any General Meeting to all members via Society email and such notice shall include details of any elections to be held.
    \item The Executive shall call further meetings either at its own initiative or at the request of 10\% of the membership or the request of the Societies Officer.
    \item At Executive committee meetings only the Society Executive committee may vote on matters concerning the operation of the society.
    \item All motions proposed during a general meeting should have a copy made of and stored in a repository for at least 5 years after date of proposal. This should be stored alongside a) its date of proposal and b) the outcome of its vote. All motions and additional information in the above repository should be made accessible to all members of the society.
\end{enumerate}

\section{Elections}
\begin{enumerate}
    \item Elections shall be held online or at a quorate general meeting in line with SU by-laws.
    \item Votes will be counted using a Single Transferrable Vote electoral system.
    \begin{role}{Shared Office}
        \item President, Treasurer, Sports Officer, and Events Officer shall be filled by one (1) seat. Secretary, Welfare, and Publicity shall be filled by up to two (2) seats. Academic, Gaming, and Social shall be filled by up to three (3) seats. Technical Officer shall be filled by up to four (4) seats. Additional positions should be filled in accordance with 6.3.2.
        \item Further officers shall be selected by taking subsequent positions in the STV process, until the required number of officers is reached.
        \begin{enumerate}
            \item This selection cannot pass a vote to Re-Open Nominations.
        \end{enumerate}
        \item Should a position not reach its quota as defined in 6.3.1, the incoming exec should treat the empty position as a vacant slot in accordance with Warwick SU's regulation 9 on vacant slots.
        \item A maximum of two (2) Members may run together in one seat, with the exception of Fresher Representative positions, which must be held by a single (1) individual.
        \item If multiple members run together, they will be considered as 1 candidate in regard to shared office and as such each member will be considered 1/nth of a full vote where n is the number of people running jointly.
        \item If a Member of a shared seat vacates their position on the Executive Committee — whether by resignation, a Vote of No Confidence, or any other means — the remaining Member shall assume full responsibility for the seat, as if it had originally been held by a single (1) individual.
    \end{role}
    \item Any amendments to the constitution must be made by the end of term 3.
    \item The renewed Constitution, with up-to-date signatures, must be sent to the Societies Coordinator before the end of term 3.
\end{enumerate}

\end{document}
